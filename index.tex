\documentclass[12pt, 
	a4paper, 
	oneside, 
	leqno
]{scrreprt}

\usepackage[T1]{fontenc}
\usepackage[utf8]{inputenc}
\usepackage[ngerman]{babel}
\usepackage[left=2cm, right=2cm, top=2.5cm, bottom=3.5cm]{geometry}
\usepackage{amssymb}
\usepackage{amsfonts}
\usepackage{amsmath, cancel}
\usepackage{thmtools}
\usepackage{url}
\usepackage{color}
\usepackage{xcolor}
\usepackage{listings}
\usepackage{hyperref}

\definecolor{lightgray}{rgb}{.9,.9,.9}
\definecolor{darkgray}{rgb}{.4,.4,.4}
\definecolor{purple}{rgb}{0.65, 0.12, 0.82}

\lstdefinelanguage{JavaScript}{
  keywords={typeof, new, true, false, catch, function, return, null, catch, switch, var, if, in, while, do, else, case, break},
  keywordstyle=\color{blue}\bfseries,
  ndkeywords={class, export, boolean, throw, implements, import, this},
  ndkeywordstyle=\color{darkgray}\bfseries,
  identifierstyle=\color{black},
  sensitive=false,
  comment=[l]{//},
  morecomment=[s]{/*}{*/},
  commentstyle=\color{purple}\ttfamily,
  stringstyle=\color{red}\ttfamily,
  morestring=[b]',
  morestring=[b]"
}

\colorlet{punct}{red!60!black}
\definecolor{background}{HTML}{EEEEEE}
\definecolor{delim}{RGB}{20,105,176}
\colorlet{numb}{magenta!60!black}

\lstdefinelanguage{json}{
    basicstyle=\tiny,
    numbers=left,
    numberstyle=\scriptsize,
    stepnumber=1,
    numbersep=8pt,
    showstringspaces=false,
    breaklines=true,
    frame=lines,
    backgroundcolor=\color{background},
    literate=
     *{0}{{{\color{numb}0}}}{1}
      {1}{{{\color{numb}1}}}{1}
      {2}{{{\color{numb}2}}}{1}
      {3}{{{\color{numb}3}}}{1}
      {4}{{{\color{numb}4}}}{1}
      {5}{{{\color{numb}5}}}{1}
      {6}{{{\color{numb}6}}}{1}
      {7}{{{\color{numb}7}}}{1}
      {8}{{{\color{numb}8}}}{1}
      {9}{{{\color{numb}9}}}{1}
      {:}{{{\color{punct}{:}}}}{1}
      {,}{{{\color{punct}{,}}}}{1}
      {\{}{{{\color{delim}{\{}}}}{1}
      {\}}{{{\color{delim}{\}}}}}{1}
      {[}{{{\color{delim}{[}}}}{1}
      {]}{{{\color{delim}{]}}}}{1},
}


\lstset{
   language=JavaScript,
   backgroundcolor=\color{lightgray},
   extendedchars=true,
   basicstyle=\footnotesize\ttfamily,
   showstringspaces=false,
   showspaces=false,
   numbers=left,
   numberstyle=\footnotesize,
   numbersep=9pt,
   tabsize=2,
   breaklines=true,
   showtabs=false,
   captionpos=b
}

\begin{document}

\chapter{Data Science - Christoph Stach}

Für unser Projekt im Modul Data Science haben wir uns darauf geeinigt eine Explorative Studie den Kurznachrichtendienst Twitter zu machen.
Dabei hatte jeder Projektteilnehmer eigene Ideen und Vorstellungen was er analsieren wollte.
Eine genaue definition der einzelnen Ziele stand nicht von Anfang an fest und hat sich im Laufe der Projektarbeit ergeben.
Die Datenbasis mit der jeder Projektteilnehmer arbeitet, sollte jedoch die gleiche sein und sich auf alle Tweets im Sammlungszeitraum in Berlin beziehen.

\section{Erkundung}

In der Erkundungsphase habe ich mich darauf Beschränkt bereits getätigte Forschungsprojekte in diesem Feld zu finden.
Dabei habe ich ein sehr interresante Internetseite mit Forschungsergebnissen gefunden.
\\
\\
\url{http://www.ling.uni-potsdam.de/~scheffler/twitter/}
\\
\\
Auf Grund dieser wissenschaftlichen Arbeit hatte ich die Idee Tweets pro Wochentag und pro Stunde am Tag darzustellen.
Außerdem wollte ich die Vorkommen von Hashtags Zählen um daraus gegebenenfalls auftretende Ergeignisse abzuleiten.
Mein Ziel war alles möglichst dynmamisch in Form einer Internetseite in einer optisch ansprechende Form darzustellen.
Es sollte auch möglich sein den Auswertungszeitraum selbst wählen zu können.

\section{Datenbeschaffung}

Twitter stellt unterschiedliche APIs\footnote{Application Programmable Interface} zur Verfügung. 
Dazu gehören unterschiedliche REST\footnote{Representational State Transfer} APIs, Streaming APIs, Webhook APIs und eine Ads API, sowie APIs für kommerzielle Zwecke.
\\
\subsection{REST APIs\footnote{\url{https://dev.twitter.com/rest/public}}}

Mit den REST Api können einmalige Anfragen an Twitter gesendet werden. Es handelt sich um Pull APIs.
Es wird eine Anfrage an den Twitter Server gesendet und man bekommt sofort ein Ergebnis zurückgeliefert.
Es lassen sich zum die Tweets einer Timeline einer Organisation oder eines Benutzers abfragen.
Außerdem lassen sich alle Informationen zu einem Benutezer abfragen. Zusätzlich kann über diese API z.B. getweetet werden.
\\
\\
Weitere detailierte Informationen findet man unter .

\subsection{Streaming APIs\footnote{\url{https://dev.twitter.com/streaming/overview}}}

Hierbei handelt es sich um eine andere Art von API. Eine Push API. Man verbindet sich zu dieser API und die Verbindung bleibt dauerhaft bestehen
bis Sie vom Benutzer oder vom Twitter Server abgebrochen wird. Solange eine Verbindung besteht sendet diese APIs in Echtzeit alle auftretenden Tweets.
Zum Zeitpunkt der Erstellung dieser Dokumentation gibt es die \emph{Public streams API}, \emph{User streams API}, \emph{Site streams API}.
\\
\\
Für meine Zwecke ist die \emph{Public streams API} interresant, diese ist laut Twitter für Data mining geeinigt ist.

\subsection{Webhook APIs\footnote{\url{https://dev.twitter.com/webhooks}}}

Laut Twitter ist diese API beta phase und wird zukünfig die \emph{User streams API} und die \emph{Site streams API} ersetzen.

\subsection{Ads API\footnote{\url{https://dev.twitter.com/ads}}}

Die Ads API ist für das Verwalten und Schalten von Werbeanzeigen auf Twitter und für unsere Zwecke nicht interresant.

\subsection{Entwicklung des Data Miners}



\section{Datenpräparation}

\section{Exploration / Modell-Planung}

\section{Model-Erstellung}

\section{Interpretation}

\section{Veröffentlichung}



\section{Operationalisierung}


\end{document}

